\documentclass[UTF8]{article}
\usepackage{ctex}
\usepackage{geometry}
\usepackage{float}
\usepackage{graphicx}
\usepackage{listings}
\usepackage{subfigure}
\usepackage{amsmath}
\geometry{a4paper,left=2cm,right=2cm,top=2cm,bottom=2cm}
\title{转速测量实验}
\author{能动A71 宋德培 2174110112}

\renewcommand{\thesection}{\Roman{section}}
\renewcommand{\thesubsection}{\arabic{section} .\arabic{subsection}}
\usepackage{xcolor}
\lstset{
	numbers=left, 
	numberstyle= \tiny, 
	keywordstyle= \color{ blue!70},
	commentstyle= \color{red!50!green!50!blue!50}, 
	frame=shadowbox, % 阴影效果
	rulesepcolor= \color{ red!20!green!20!blue!20} ,
	escapeinside=``, % 英文分号中可写入中文
	xleftmargin=2em,xrightmargin=2em, aboveskip=1em,
	framexleftmargin=2em
} 
\begin{document}
	\maketitle
	\section{实验目的}
	\begin{enumerate}
	\item 了解光电、磁电和霍尔式三种转速传感器的基本结构和工作原理。
	\item 认识光电、磁电和霍尔式三种转速传感器的输出波形,以及不同转速下三种传感器输出波形的变化规律。
	\item 分别测出500转/分、1000转/分、2000转/分和3000转/分下三种传感器的输出波形(画一个完整脉冲波形),记录输出电压。
	
	\end{enumerate}
	
	\section{实验原理}
	\subsection{光电式转速传感器}
	基于光电子元件的光电效应,当具有一定能量的光子投射到某些物质表面时,具有辐射能量的微粒将透过受光物质的表面层,赋予这些物质的电子以附加能量,或者改变物质的电阻大小,或者产生电动势,从而实现光电转换过程。
	\subsection{磁电式转速传感器}
	利用电磁感应原理,将输入的运动速度转换成感应电势输出。根据电磁感应定律,当$W$匝线圈在均衡磁场中运动时,设穿过线圈的磁通为$\Phi$,则线圈内感应电势为
	\[
	e = -W\frac{d\Phi}{dt}
	\]
	根据这一原理可制成恒磁通或者变磁通传感器。
	
	本实验采用变磁通结构,动铁心的转动使气隙和磁路磁阻变化,引起磁通改变从而感应出电势,将单位时间内脉冲数除以齿数则表示旋转频率。
	\subsection{霍尔式转速传感器}
	基于某些材料的霍尔效应。在与磁场垂直的半导体薄片中通电流$I$,则多数载流子受到洛伦兹力作用向一侧偏转,造成电子积累,在两个侧面建立起电场$E$,因此电子又受到该电场作用$F_E$,当这个力与洛伦兹力相等,建立起动平衡,两个侧面建立起电位差$U_H$,称为霍尔电压。对特定材料这是一个常数。
	\[
	U_H=K_HIB
	\]
	其中$K_H$为霍尔灵敏度。
	\section{操作要点}
	1.	对于CGD-2000转速实验仪,当电压调节到超过20V(因在20V以下,电机不能转动)时,电机才开始启动。
	
	2.	为了延长电机的寿命,设定了保护转速(3800转/min),电机一旦超过此转速,仪表就会保护,切断电机电源,CGD-2000转速实验仪会发出蜂鸣声。需要再次启动电机时,先将调速旋钮回零,按下电源开关按钮,再次重新启动试验箱即可。
	
	3.	记录电压输出值时,注意区分出输出值的正负。
	
	4.	若每转有多个脉冲信号,只需画出一个脉冲信号的图形。
	
	5.	选择用于检测转速的传感器,按下相对应的按键,与之相对应的指示灯点亮(绝对不允许同时按下2个或3个按键,会损坏仪器)。顺时针转动调速旋钮,电机缓慢启动。转速直接显示到转速表上,示波器也随之显示相应的波形
	
	6.	实验过程中,同时将散热的开关打开,避免电机散发的热量影响试验箱的正常工作。
	
	
	\section{实验结果分析}
	\noindent\textbf{1.光电式、霍尔式和磁电式传感器的各输出是何波形?}
	
	光电式传感器输出波形为矩形波,当被遮挡时输出为高电平,当无遮挡时输出为低电平。
	
	霍尔式传感器输出波形也是矩形波,但当气隙经过传感器时输出为高电平,当转子经过时输出为低电平。
	
	磁电式传感器输出波形为连续的周期性震荡,其明显含有正弦基波并伴随有谐波。
	
	\noindent\textbf{2.三种传感器输出峰值电压与转速的高低有何种规律?}
	
	霍尔式和光电式传感器输出的电压峰值随转速变化基本不变。磁电式传感器输出电压峰值与转速成正比。
	
	\noindent\textbf{3.分析磁电式传感器为何不能进行低转速测量?}
	
	由电磁感应定律
	\[
	e=-W\frac{d\Phi}{dt}
	\]
	因此当转速较低时,磁通变化率小,感应电势低,一方面小信号不容易精确采集,另一方面输出的信号易受到噪声干扰,信噪比低。
	
	\noindent\textbf{4.如果在转速较低时需使用磁电式传感器测量,可采用哪些措施?}
	
	为了使得输出电压信号增大,可以减小磁阻,例如减小传感器与被测物体距离以减小气隙磁阻,将被测物体材料改为磁导率更好的金属。还可以使用前置放大器放大电压信号,增加滤波器滤除杂波干扰等。
	
	\section{原始数据记录}
	附录为原始数据记录\footnote{在实际实验过程中发现在较高转速下磁电式传感器输出电压不再线性增大,原因是当被测物体转速超过磁电式传感器测量范围时,磁路损耗过大,将使得输出电势饱和。}
	
\end{document}